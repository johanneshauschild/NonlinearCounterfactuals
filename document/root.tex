\documentclass[10pt,a4paper]{article}
\usepackage[english]{babel}
\usepackage[utf8]{inputenc}
\usepackage{url}
\usepackage{csquotes}
\usepackage{amsmath}
\usepackage{amssymb}
\usepackage{isabelle,isabellesym}

\usepackage{color}

\usepackage[top=3cm,bottom=4.5cm]{geometry}

%tikz libs
\usepackage{graphicx}
\usepackage{tikz}
\usetikzlibrary{positioning}
\usetikzlibrary{shapes.geometric, arrows, arrows.meta}
\usetikzlibrary{calc,decorations.pathmorphing,shapes}

\definecolor{keyword}{RGB}{0,153,102}
\definecolor{command}{RGB}{0,102,153}
\isabellestyle{tt}
\renewcommand{\isacommand}[1]{\textcolor{command}{\textbf{#1}}}
\renewcommand{\isakeyword}[1]{\textcolor{keyword}{\textbf{#1}}}

% this should be the last package used
\usepackage{pdfsetup}

% urls in roman style, theory text in isabelle-similar-similar type-writer
\urlstyle{rm}
\isabellestyle{tt}

\title{ \textbf{Nonlinear Counterfactuals in Isabelle/HOL}}
\date{\today}
\author{Johannes Hauschild}

\begin{document}

\maketitle

\begin{abstract}
  This thesis investigates the semantics of Lewis' counterfactual operators, as well as their improvements by Finkbeiner and Siber. Moving on, we give a further improvement.

  To this end we formalise the counterfactual operators under investigation in a logic CFL defined over world dependent Kripke structures. 
  Problematic assumptions about these Kripke structures are necessary for Lewis' operators to obtain their (likely) intended semantics.
  We validate that Finkbeiner and Sibers operators enable a relaxation of these assumptions.
  Furthermore, we define counterfactual operators which again allow for a further relaxation.
  
  Finally, we analyse the relationship between CFL and CTL* using a translation function.
  Results show that using the translation function, CTL* and CFL are not comparable in terms of their expressive power.
  
  Our findings are backed by a formalisation in Isabelle/HOL.
\end{abstract}

% include generated text of all theories
\input{session}

\phantomsection
\addcontentsline{toc}{section}{References}
\bibliographystyle{alphaurl}
\bibliography{root}


\end{document}
